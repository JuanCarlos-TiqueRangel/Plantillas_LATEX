\chapter{Nombre del capitulo}

Los capítulos son las principales divisiones del documento. En estos se desarrolla el tema del documento. Cada capítulo debe corresponder a uno de los temas o aspectos considerados dentro del proyecto que dio lugar al trabajo de grado.  El interlineado de texto en capítulos es de 1.15.

\section{Subtítulos nivel 2}
Toda división o capítulo, a su vez, puede subdividirse en otros niveles y sólo se enumera hasta el tercer nivel. Los títulos de segundo nivel se escriben con minúscula al margen izquierdo y sin punto final, están separados del texto o contenido por un interlineado posterior de 10 puntos y anterior de 20 puntos (tal y como se presenta en la plantilla).

\subsection{Subtítulos nivel 3}

De la cuarta subdivisión en adelante, cada nueva división o ítem puede ser señalada con viñetas, conservando el mismo estilo de ésta, a lo largo de todo el documento.\\

\begin{itemize}
\item En caso que sea necesario utilizar viñetas, use este formato (viñetas cuadradas).
\end{itemize}

Las subdivisiones, las viñetas y sus textos acompañantes deben presentarse sin sangría y justificados. 

\begin{figure}[h]
\centering
\includegraphics[scale=0.5]{figures/logo_universidad.png}
\caption{ejemplo imagen}
\end{figure}


 

